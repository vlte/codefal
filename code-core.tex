% file code-code.tex
%%% Document history
% YYYY-MM-DD -- auteur : descr modification
% 2008-12-30 -- ViooLooute : transcription du code du 120 en LaTeX.
% 2009-03-09 -- ViooLooute : extraction 
%
%%%


\setcounter{secnumdepth}{0} % suppress numbering starting a the section level

\section*{Historique du code}
D'après les décisions des grand maîtres lors :
\begin{description}
 \item[Centenaire de la faluche], Reims les 25 et 26 juin 1988
 \item[Etats généraux de la faluche], Dijon, les 23, 24 et 25 Juin 1989
 \item[Convention nationale de la faluche], Lille, les 22, 23, et 24 juin 1990
 \item[Congrès nationaux de la Faluche] :

 \item[Clermont-Ferrand] les 23, 24, et 25 juin 1991 (103)
 \item[Poitiers] les 24, 25 et 26 juin 1992 (104)
 \item[Nancy] les 3, 4 et 5 juillet 1993 (105)
 \item[Toulouse] les 1, 2 et 3 juillet 1994 (106)
 \item[Paris] les 30 juin, 1 et 2 juillet 1995 (107)
 \item[Orléans] les 28, 29 et 30 Juin 1996 (108)
 \item[Montpellier] les 27, 28, 29 Juin 1997 (109)
 \item[Reims] les 3, 4 et 5 Juillet 1998 (110)
 \item[Grenoble] les 25, 26 et 27 Juin 1999 (111)
 \item[Lille] les 30 juin, 1 et 2 juillet 2000 (112)
 \item[Poitiers] les 29, 30 juin, et 1 juillet 2001 (113)
 \item[Paris] les 28, 29 et 30 juin 2002 (114)
 \item[Bordeaux] les 27, 28, 29 juin 2003 (115)
 \item[Toulouse] les 2, 3, 4 juillet 2004 (116)
 \item[Lyon] les 8, 9, 10 juillet 2005 (117)
 \item[Onzain] les 7, 8, 9 juillet 2006 (118)
 \item[Montpellier] les 29, 30 juin et 1 juillet 2007 (119)
 \item[Reims] les 4, 5, 6 juillet 2008 (120)


\end{description}

\chapter*{Code}
         Notre béret d'étudiant fut ramené de Bologne, en juin 1888, par la délégation
française d'un congrès international d'étudiants qui, jalouse de voir le chapeau façon
Louis XI des étudiants italiens et la casquette plate des étudiants belges et allemands,
décida d'avoir une coiffure spécifique aux étudiants français. Elle adopta le béret de
velours des habitants de la région bolognaise, en souvenir du congrès qui fut, paraît-il,
magnifique.

\section{Article I}
        La faluche est la coiffe traditionnelle des étudiants de France. Elle a remplacé la
toque datant du Moyen-Âge. Les étudiants français l'ont ramenée de Bologne, lors d'un
congrès international d'étudiants où ils adoptèrent le béret de velours des habitants de la
région bolognaise, le 12 juin 1888.

\section{Article II}
         La faluche est portée de nos jours dans certaines facultés et écoles, d'une manière
habituelle ou à titre exceptionnel, lors de manifestations estudiantines. On ne l'enlève pas,
même devant un professeur, sauf s'il a le rang de recteur. Elle ne se réclame d'aucune
appartenance politique ou religieuse.

\section{Article III}
         Escholier, il est interdit de faire de la faluche une succursale de monoprix. Elle doit
être le parchemin qui s'enorgueillit toujours de nouvelles richesses de l'histoire de
l'étudiant.
         A ce titre un Grand Maître (peu importe lequel) peut décider de faire enlever un
insigne qu'il juge non représentatif de l'étudiant.

\section{Article IV}
          L'étudiant doit se conformer à ces dispositions assez larges pour permettre toute
fantaisie, assez strictes pour réaliser l'uniformité.

\section{Article V}
La faluche comporte deux parties :
\begin{enumerate}
	\item Le ruban circulaire avec ses emblèmes (cursus estudiantin).
	\item Le velours noir avec ses rubans et insignes.
\end{enumerate}

\section{Article VI}
\label{Article VI}
         Le ruban circulaire est à la couleur de la discipline principale. Pour 
les disciplines secondaires (double inscription) on placera, sur le bord 
supérieur du ruban de la discipline principale, un ruban plus mince aux couleurs
de cette (ou ces) discipline(s) annexe(s).
\newline

         Voici les couleurs nationalement adoptées, basées à l'origine sur celles des toges
doctorales et professorales des universités françaises :

\subsection{Rubans circulaires de velours}
\begin{center}
\begin{tabularx}{1.2\textwidth}{XX}
% \begin{tabularx}{1.2\textwidth}{|X|X|}
% \hline
% \multicolumn{2}{|c|}{Circulaires de velours}\\
% \hline
% \hline

Chirurgie dentaire & Velours violet\\ 
Médecine           & Velours rouge\\ 
Ostéopathie        & Velours bleu\\ 
Paramédical        & Velours rose\\ 
Pharmacie          & Velours vert \\ 
Prépas santé       & Velours marron\\ 
Sage femme         & Velours fuchsia\\
Vétérinaire        & Velours bordeaux\\
% \hline
\end{tabularx}
\end{center}

\subsection{Rubans circulaires de satin}
\begin{center}
\begin{tabularx}{1.2\textwidth}{XX}
% \begin{tabularx}{1.2\textwidth}{|X|X|}
% \hline
% \multicolumn{2}{|c|}{Circulaires de satin}\\
% \hline
% \hline
Administration Économique et Sociale &  Satin vert clair\\ 
Architecture, Beaux-Arts, Cinéma, Théâtre &  Satin bleu\\ 
Droit & Satin rouge \\ 
Écoles d'ingénieurs &  Satin noir et bleu\\ 
Écoles de commerce &  Satin rouge et vert\\ 
IUT, BTS &  Satin blanc\\ 
IUP &  Satin aux couleurs de la discipline\\ 
Filières sportives & Satin vert foncé \\ 
Lettres, Sciences Humaines et Sociales &  Satin jaune\\ 
Musique et Musicologie & Satin argenté \\ 
\OE nologie & Satin saumon \\ 
Prépas (Taupes, Khâgne, ...) &  Satin marron\\ 
Sciences & Satin violet\\ 
Sciences économiques, Gestion & Satin orange\\ 
Sciences politiques & Satin rouge et bleu\\
% \hline
\end{tabularx}
\end{center}

\section{Article VII}
Sur le ruban circulaire doivent figurer :
\subsection{Le baccalauréat}

\begin{center}
\begin{tabularx}{\textwidth}{X c X}
Bac L   & $\varphi$ & (Phi)\\
Bac ES  & $\beta$   &(Béta)\\
Bac S spé math ou physique & $\varepsilon$ &(Epsilon)\\
Bac S spé bio  & $\varphi\varepsilon$ & (Phi Epsilon)\\
Autres bacs & \multicolumn{2}{l}{Lettre ou initiale correspondantes} \\
Bac international &  \multicolumn{2}{l}{\og I \fg~  minuscule après bac principal}\\
\end{tabularx}
\end{center}

Note : Pour les capacitaires, l'emblème du baccalauréat sera remplacé par la lettre \og C \fg~
majuscule.

\subsection{L'emblème de la discipline}

\begin{center}
\begin{tabularx}{\textwidth}{l X}

Administration économique et sociale  & \og AES\fg \\
Agro-alimentaire                      & Fourchette et Épi de blé croisés\\
Architecture                          & Équerre et compas\\
Archéologie                           & Tête de sphinx\\
Beaux-arts, Cinéma                    & Palette et pinceau\\
B.T.S.                                & \og BTS\fg \\
Chirurgie dentaire                    & Molaire\\
Prépas (Taupes, Khâgne, ...)          & Chouette à deux faces\\
Droit                                 & Balance\\
Écoles d'ingénieurs                   & Étoile et foudre\\
Écoles de commerce                    & Caducée mercure\\
Filières sportives                    & Coq\\
Géographie                            & Globe\\
Histoire                              & Casque de Périclès\\
I.U.T.                                & \og IUT\fg \\
I.U.P.                                & \og IUP\fg\\
Kiné                                  & Caducée mercure\\
Lettres, Langues                      & Livre ouvert et plume\\
Médecine                              & Caducée médecine\\
Musique et Musicologie                & Lyre\\
\OE nologie                           & Grappe de raisin\\
Ostéopathie                           & Sphénoïde\\
Paramédical                           & Ciseaux\\
PCEM1                                 & Tête de mort croisée sur fémurs\\
Pharmacie                             & Caducée pharmacie\\
Psychologie                           & $\Psi$ (Psi)\\
Sage femme                            & Croix d'Ânkh\\
Social                                & Initiales de la filière\\
Sociologie                            & Grenouille\\
Sciences                              & Palmes croisées de chêne et de laurier et
                                     Initiales de la filière\\
Sciences économiques, Gestion         & Caducée mercure\\
Sciences politiques                   & Parapluie\\
Théâtre                               & Masque de comédie\\
Vétérinaire                           & Tête de cheval\\
\end{tabularx}
\end{center}


\subsection{Les étoiles et palmes}

\subsubsection*{Étoiles}
\begin{itemize}
\item Une étoile dorée par année d'études (se place en début d'année).
	\begin{itemize}
	\item Une large palme placée à coté de l'étoile de l'année pour le major de la promotion.
	\item Les étoiles de disciplines annexes seront de taille plus petite.
	\end{itemize}

\item Une étoile argentée pour les années redoublées.

\item Un petit  \og E\fg~ remplacera l'étoile de l'année obtenue par équivalence.


\item La première étoile sera placée sur un petit ruban disposé de biais, de couleur bleue
pour les facultés ou établissements d'état, blanche pour les facultés catholiques, écoles ou
instituts privés. Tout changement d'une université d'État à privée ou inversement sera
signifié par un nouveau ruban sous l'étoile de la nouvelle première année.
\end{itemize}

\subsubsection*{Palmes}
\begin{itemize}
\item Une simple palme à la fin de chaque cycle (s'il n'y a pas de diplôme correspondant)
\item Une double palme croisée de lauriers pour chaque diplôme obtenu ( DEUG, DUT,
BTS, Licence, Maîtrise, diplôme universitaire, diplôme d'état, \ldots).
\item à coté de l'étoile de l'année :
	\begin{itemize}
	\item  Une tête de vache pour tout échec aux examens de la première session et
	réussite à la session de rattrapage.
	\item Une tête de mort pour abandon d'une discipline.
	\end{itemize}

\item Une quille pendra du ruban à l'endroit du cursus où aura été effectué le service
militaire.

\item Un drapeau du pays sera placé sous l'étoile de l'année d'étude si elle se déroule à
l'étranger.

\end{itemize}

\section{Article VIII}
\label{Article VIII}
Le velour peut comporter :
\subsection{Les insignes}
\begin{itemize}
\item Des associations étudiantes.
\item Des congrès auxquels vous avez participé, à condition que ceux-ci soient constitués
de matériaux nobles (tissus, métaux).
\item Des villes où vous avez séjourné pour motifs étudiants.
\item De tous les établissements scolaires auxquels vous avez appartenu.
\item De tous les clubs auxquels vous avez appartenu et auxquels vous appartenez.
\end{itemize}

\subsection{Votre devise}
         En grec, en latin, en français, en hébreu, en langue régionale, en patois\ldots s'inscrit
en toutes lettres dans la langue correspondante, sur le velours du frontal à l'occipital à la
gauche du ruban de province natale.

\subsection{Vos armes personnelles}

\subsection{Les symboles}
\begin{center}
\begin{tabularx}{\textwidth}{X X}
% \hline
\multicolumn{2}{l}{Insignes :}\\
% \hline 
\\
Ancre                            & Amour de la navigation\\
Chameau                          & à l'endroit, célibataire\\
                                 & à l'envers, c\oe ur pris\\
Cochon                           &  Bizuté : à l'envers, l'a été\\
                                 & à l'endroit, ne l'a pas été  \\
Épi de blé croisé d'une faucille & Chanceux aux examens\\
Épi de blé                       & Radin\\
Fer à cheval                     & Chanceux\\
Feuille de vigne                 & Perte de la virginité masculine\\
Fourchette                       & Amour des plaisirs de la table\\
Grappe de raisin                 & Amour du bon vin\\
Lyre                             & Amour de la musique\\
Palette vernie                   & Amour de la peinture\\
Pendu                            & Marié(e)\\
Plume                            & Amour de la littérature\\
Rose                             & Perte de la virginité féminine\\
Sphinx                           & Polyglotte\\
Squelette                        & à l'endroit, amour de l'anatomie\\
                                 & à l'envers, amour de l'anatomie du sexe opposé \\
                                 & à l'envers avec une pointe de diamant entre les
                                   jambes, homosexuel\\
\end{tabularx}
\end{center}

\begin{center}
\begin{tabularx}{\textwidth}{X X}
% \hline
\multicolumn{2}{l}{Insignes décernés par le Grand Maître :}\\
% \hline 
\\
Bacchus                   & Dignité dans l'ivresse (retournable et retirable)\\
Bouteille de bordeaux     & Cuite certifiée (possibilité de coefficient
                            multiplicateur)\\
Bouteille de Champagne    & Coma éthylique certifié\\
Chouette                  & Oiseaux de nuit\\
Clé de sol                & Digne chanteur de paillarde\\
Coq                       & Grande gueule, sachant l'ouvrir\\
Cor de chasse             & Grand chasseur devant l'Eternel\\
Fourchette sur ruban bleu & Cordon bleu\\
Hache                     & Prise de guerre (acte exceptionnel, à
                            ne pas confondre avec vandalisme)\\
Poule                     & Fille très chaude\\ % Serait : Garçon ou fille très chaud.
Singe                     & Quémandeur d'insignes, empêche
                            de recevoir tout autre insigne.\\
Sou troué                 & Nuit passée au poste pour motif étudiant\\

\end{tabularx}
\end{center}

\begin{center}
\begin{tabularx}{\textwidth}{X X}

% \hline
\\
\multicolumn{2}{l}{Insignes décernés par le ou la partenaire :}\\
% \hline 
\\
Épée        & Fin baiseur\\
Flèche      & Éjaculateur précoce (décernée par la fille)\\
Lime        & Acte laborieux et difficile\\
Pensée      & Experte\\

\end{tabularx}
\end{center}

\begin{center}
\begin{tabularx}{\textwidth}{X X}

% \hline
\\
\multicolumn{2}{l}{Insigne placé sur le ruban d'association :}\\
% \hline 
\\
Abeille                       &Travail et minutie associatif\\
\end{tabularx}
\end{center}


\section{Article IX}
        Lors d'une garde assurée par un étudiant en santé tout passage de vie à trépas sera
sanctionné par une faux placée sur le velours noir.

\section{Article X}
 Sur le velours figurent aussi les rubans supérieurs de gauche à droite, du frontal à l'occipital:

\subsection{Ruban de ville de faculté}
          Ruban perpendiculaire à celui de l'association, aux couleurs de la ville de faculté,
surmonté d'un écusson en toile.
          L'étudiant changeant de ville de faculté (France ou étranger) placera parallèlement
et en arrière du précédent, un ruban surmonté d'un écusson, aux couleurs de sa nouvelle
ville universitaire, et l'année de changement en chiffres.

\subsection{Le ruban d'association}
Rubans aux couleurs de l'association précisant la place occupée par l'étudiant.


\subsection{Pour les représentants d'association}


\begin{enumerate}[a -- ]
\item Pour les membres du conseil d'administration et du bureau d'association régionale : 
un ruban aux couleurs de la ville d'élection.
 
\item Pour les élus et délégués au plan national :
	\begin{itemize}
	\item pour les membres du conseil d'administration des associations, unions,
	fédérations nationales : un ruban tricolore.
	\item pour les membres de bureau : un ruban tricolore avec un filet blanc de chaque
	coté.
	\item pour le président : un ruban tricolore avec une bande blanche de chaque coté.
	\end{itemize}

\item Pour les élus ou délégués au plan européen : 
les couleurs de l'Europe.

\item Pour les élus ou délégués au plan international : 
les couleurs de l'O.N.U.

\end{enumerate}
Le délégué placera le ruban de plus haut grade.


\subsection{Élus au conseil d'UFR, conseils d'Université, conseils régionaux et nationaux}
L'élu placera, du frontal à l'occipital, un ruban de couleur jaune dont l'extrémité
occipital sera laissée libre.
          Sur ce pendentif seront placées, à raison d'un insigne par mandat :

\begin{description}
\item[grenouille argentée]: pour les élus au conseil d'UFR
\item[grenouille dorée]: pour les élus au conseil d'Université
\item[tortue argentée]: pour les élus au CROUS
\item[tortue dorée]: pour les élus au CNOUS
\item[les initiales] des autres conseils (CNESER, OVE, ...)
\item[étoile dorée]: pour les VP étudiant d'université
\item[étoile argentée]: pour les VP étudiant d'UFR ou de CA d'IUT
\item[étoile dorée (sur un petit ruban bleu)] par mandat de délégués mutualistes 
\end{description}


\subsection{Les rubans de province et ville d'origine}
Surmontés des écussons en toile correspondants.


\section{Article XI}
         La faluche montpelliéraine se distingue par la présence de quatre crevés aux
couleurs de la discipline principale étudiée, divisant ainsi le couvre-chef en quatre parties
de velours noir égales. Ce \og privilège\fg est réservé aux seuls étudiants de l'Université de
Montpellier, en hommage à Rabelais qui fit ses études de médecine à Montpellier au
XVIème siècle. Il enseigna la médecine à Montpellier qui est la plus ancienne Université
française (XIIème siècle).

Les disciplines médicales ont conservé les couleurs traditionnelles.
\begin{center}
\begin{tabularx}{\textwidth}{l X}
   Médecine           & 4 crevés bordeaux \\
   Pharma             & 4 crevés verts\\
   Chirurgie-dentaire & 2 crevés bordeaux et 2 crevés violets\\
   Paramédicale       & 4 crevés roses\\
\end{tabularx}
\end{center}


\section{Article XII}
 Tout étudiant ayant, au cours de sortie, repas ou soirée, tiré un coup en bonne et
due forme, devra mettre à l'intérieur de sa faluche :
\begin{description}
      \item  [une carotte], signe de son acte valeureux et digne du grand baisouillard qu'il est.
      \item [un poireau] pour une pipe dûment accomplie.
      \item [un navet], pour l'enculage.
      \item [une betterave], pour un cunnilingus.
 \end{description}

 Ceci sous l'\oe il attentif des anciens, dignes contrôleurs des actes accomplis. Ils
contrôleront en particulier que l'étudiant était sorti couvert.
 Pour tout dépucelage, il aura droit, suivant l'endroit, à deux légumes placés en X.
 En espérant voir les faluches se transformer durant les années estudiantines, en de
véritables potagers.

\section{Article XIII}

 Dans chaque faculté ou école, il sera procédé à l'\textbf{élection} d'un Grand Maître, choisi
par les faluchards au vu de ses mérites, et dont la tâche principale sera de veiller à
l'application des principes de base qui régissent le port de la faluche.

 Sa distinction est une croix du mérite émaillée blanc soutenue par un ruban aux
couleurs de la discipline. Lui seul est en mesure de décerner le Bacchus, l'abeille, la
chouette, le coq, \ldots  (voir article VIII).

          Il doit en outre certifier, avec le concours de trois autres porteurs de faluches,
l'acquisition de bouteilles.

\section{Article XIV}
 Toute faluche devra être baptisée sous l'oeil attentif du Grand Maître ou des anciens.

\section{Article XV}
 La faluche doit être portée dans toute cérémonie : repas, soirées, sorties estudiantines, \ldots


\section{Article XVI}
 Toute pucelle effarouchée ou donzelle à la jambe mutine demandant à voir le
potager particulier (intérieur de la faluche) d'un étudiant, devra comme il se doit, en passer
par les armes suivant les goûts dudit étudiant, choisissant le lieu, le jour et l'heure.
 La faluche représentant la vie d'un étudiant, toute personne mettant une faluche
ne lui appartenant pas devra subir une épreuve qu'aura choisi le propriétaire de la faluche
en question.


\section{Article XVII}
 La faluche est un attribut qui doit être respecté par tous. Des sanctions pourront être prises 
pour tout non-respect vis-à-vis d'une faluche. D'autre part, tout étudiant surpris
en train de baptiser sans être porteur d'une faluche sera lui-même considéré comme
impétrant. A ce titre, il encourt les pires sévices \ldots
\begin{flushright}
Les Grands Maîtres,
\end{flushright}


\chapter*{Annexes}
% 
%% FIGURE

\section*{Le ruban circulaire (art. VI et VII)}
\begin{figure}[h]
\centering
	\includegraphics*[width=0.80\textwidth]{front}
   \caption{Vue de face}
   \label{FIG:circulaire}
\end{figure}


Légende de la figure \ref{FIG:circulaire} :
\begin{description}
\item[1]: Armes personnelles : d'avant en arrière
	\begin{itemize}
	\item Insigne de la discipline
	\item BAC + année d'obtention
	\item Initiales en majuscules
	\end{itemize}
\item[2] : Étoile de bizuth sur ruban en biais
\item[3] : Une étoile par année
\item[4] : Palme de fin de 1er cycle
\end{description}


\section*{Le velour}
\begin{figure}
\centering
	\includegraphics*[width=0.90\textwidth]{top}
   \caption{Vue du dessus}
   \label{FIG:velour}
\end{figure}

Légende de la figure \ref{FIG:velour} :
\begin{description}
\item[A]: Côté officiel - Titres et insignes officiels (art. VIII \S 1 et X \S 1 à 3)

	\begin{description}
	\item[a] Ruban aux couleurs de la ville de faculté surmonté de l'écusson en toile
	\item[b] Ruban d'association :
		\begin{description}
		\item[b1] : Simple adhérent de l'association
		\item[b2] : Élu ou coopté au sein de l'association
		\end{description}	
	\item[c] Ruban des représentants d'associations
	\item[d] Ruban de conseil d'U.F.R. ou d'université
	\end{description}

\item[B]: Côté officiel badges de congrès
\item[C]: Côté personnel (art. VIII \S 2 à 4 et X \S 4) : coq, cochon, squelette, chameau ...
	\begin{description}
	\item[e] Couleur de la province natale
	\item[f] Couleur de la ville natale
	\item[g] Écusson de la ville natale
	\item[h] Écusson de la province natale
	\end{description}
\item[D]: Côté voyage
\end{description}
